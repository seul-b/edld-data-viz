% Options for packages loaded elsewhere
\PassOptionsToPackage{unicode}{hyperref}
\PassOptionsToPackage{hyphens}{url}
%
\documentclass[
]{article}
\usepackage{amsmath,amssymb}
\usepackage{lmodern}
\usepackage{iftex}
\ifPDFTeX
  \usepackage[T1]{fontenc}
  \usepackage[utf8]{inputenc}
  \usepackage{textcomp} % provide euro and other symbols
\else % if luatex or xetex
  \usepackage{unicode-math}
  \defaultfontfeatures{Scale=MatchLowercase}
  \defaultfontfeatures[\rmfamily]{Ligatures=TeX,Scale=1}
\fi
% Use upquote if available, for straight quotes in verbatim environments
\IfFileExists{upquote.sty}{\usepackage{upquote}}{}
\IfFileExists{microtype.sty}{% use microtype if available
  \usepackage[]{microtype}
  \UseMicrotypeSet[protrusion]{basicmath} % disable protrusion for tt fonts
}{}
\makeatletter
\@ifundefined{KOMAClassName}{% if non-KOMA class
  \IfFileExists{parskip.sty}{%
    \usepackage{parskip}
  }{% else
    \setlength{\parindent}{0pt}
    \setlength{\parskip}{6pt plus 2pt minus 1pt}}
}{% if KOMA class
  \KOMAoptions{parskip=half}}
\makeatother
\usepackage{xcolor}
\usepackage[margin=1in]{geometry}
\usepackage{color}
\usepackage{fancyvrb}
\newcommand{\VerbBar}{|}
\newcommand{\VERB}{\Verb[commandchars=\\\{\}]}
\DefineVerbatimEnvironment{Highlighting}{Verbatim}{commandchars=\\\{\}}
% Add ',fontsize=\small' for more characters per line
\usepackage{framed}
\definecolor{shadecolor}{RGB}{248,248,248}
\newenvironment{Shaded}{\begin{snugshade}}{\end{snugshade}}
\newcommand{\AlertTok}[1]{\textcolor[rgb]{0.94,0.16,0.16}{#1}}
\newcommand{\AnnotationTok}[1]{\textcolor[rgb]{0.56,0.35,0.01}{\textbf{\textit{#1}}}}
\newcommand{\AttributeTok}[1]{\textcolor[rgb]{0.77,0.63,0.00}{#1}}
\newcommand{\BaseNTok}[1]{\textcolor[rgb]{0.00,0.00,0.81}{#1}}
\newcommand{\BuiltInTok}[1]{#1}
\newcommand{\CharTok}[1]{\textcolor[rgb]{0.31,0.60,0.02}{#1}}
\newcommand{\CommentTok}[1]{\textcolor[rgb]{0.56,0.35,0.01}{\textit{#1}}}
\newcommand{\CommentVarTok}[1]{\textcolor[rgb]{0.56,0.35,0.01}{\textbf{\textit{#1}}}}
\newcommand{\ConstantTok}[1]{\textcolor[rgb]{0.00,0.00,0.00}{#1}}
\newcommand{\ControlFlowTok}[1]{\textcolor[rgb]{0.13,0.29,0.53}{\textbf{#1}}}
\newcommand{\DataTypeTok}[1]{\textcolor[rgb]{0.13,0.29,0.53}{#1}}
\newcommand{\DecValTok}[1]{\textcolor[rgb]{0.00,0.00,0.81}{#1}}
\newcommand{\DocumentationTok}[1]{\textcolor[rgb]{0.56,0.35,0.01}{\textbf{\textit{#1}}}}
\newcommand{\ErrorTok}[1]{\textcolor[rgb]{0.64,0.00,0.00}{\textbf{#1}}}
\newcommand{\ExtensionTok}[1]{#1}
\newcommand{\FloatTok}[1]{\textcolor[rgb]{0.00,0.00,0.81}{#1}}
\newcommand{\FunctionTok}[1]{\textcolor[rgb]{0.00,0.00,0.00}{#1}}
\newcommand{\ImportTok}[1]{#1}
\newcommand{\InformationTok}[1]{\textcolor[rgb]{0.56,0.35,0.01}{\textbf{\textit{#1}}}}
\newcommand{\KeywordTok}[1]{\textcolor[rgb]{0.13,0.29,0.53}{\textbf{#1}}}
\newcommand{\NormalTok}[1]{#1}
\newcommand{\OperatorTok}[1]{\textcolor[rgb]{0.81,0.36,0.00}{\textbf{#1}}}
\newcommand{\OtherTok}[1]{\textcolor[rgb]{0.56,0.35,0.01}{#1}}
\newcommand{\PreprocessorTok}[1]{\textcolor[rgb]{0.56,0.35,0.01}{\textit{#1}}}
\newcommand{\RegionMarkerTok}[1]{#1}
\newcommand{\SpecialCharTok}[1]{\textcolor[rgb]{0.00,0.00,0.00}{#1}}
\newcommand{\SpecialStringTok}[1]{\textcolor[rgb]{0.31,0.60,0.02}{#1}}
\newcommand{\StringTok}[1]{\textcolor[rgb]{0.31,0.60,0.02}{#1}}
\newcommand{\VariableTok}[1]{\textcolor[rgb]{0.00,0.00,0.00}{#1}}
\newcommand{\VerbatimStringTok}[1]{\textcolor[rgb]{0.31,0.60,0.02}{#1}}
\newcommand{\WarningTok}[1]{\textcolor[rgb]{0.56,0.35,0.01}{\textbf{\textit{#1}}}}
\usepackage{graphicx}
\makeatletter
\def\maxwidth{\ifdim\Gin@nat@width>\linewidth\linewidth\else\Gin@nat@width\fi}
\def\maxheight{\ifdim\Gin@nat@height>\textheight\textheight\else\Gin@nat@height\fi}
\makeatother
% Scale images if necessary, so that they will not overflow the page
% margins by default, and it is still possible to overwrite the defaults
% using explicit options in \includegraphics[width, height, ...]{}
\setkeys{Gin}{width=\maxwidth,height=\maxheight,keepaspectratio}
% Set default figure placement to htbp
\makeatletter
\def\fps@figure{htbp}
\makeatother
\setlength{\emergencystretch}{3em} % prevent overfull lines
\providecommand{\tightlist}{%
  \setlength{\itemsep}{0pt}\setlength{\parskip}{0pt}}
\setcounter{secnumdepth}{-\maxdimen} % remove section numbering
\ifLuaTeX
  \usepackage{selnolig}  % disable illegal ligatures
\fi
\IfFileExists{bookmark.sty}{\usepackage{bookmark}}{\usepackage{hyperref}}
\IfFileExists{xurl.sty}{\usepackage{xurl}}{} % add URL line breaks if available
\urlstyle{same} % disable monospaced font for URLs
\hypersetup{
  pdftitle={Lab 2},
  pdfauthor={Seulbi Lee},
  hidelinks,
  pdfcreator={LaTeX via pandoc}}

\title{Lab 2}
\author{Seulbi Lee}
\date{2023-02-04}

\begin{document}
\maketitle

The purpose of this lab is to use color to your advantage. You will be
asked to use a variety of color palettes, and use color for its three
main purposes: (a) distinguish groups from each other, (b) represent
data values, and (c) highlight particular data points.

\hypertarget{data}{%
\subsection{Data}\label{data}}

We'll be working with the honey production data from \#tidytuesday. The
\href{https://github.com/rfordatascience/tidytuesday/tree/master/data/2018/2018-05-21}{\#tidytuesday
repo} contains the full data, but we'll work with just the cleaned up
version, using the \textbf{honeyproduction.csv} file, which is posted on
the website.

The data is in under Dataset tab of Week 4 module on Canvas.

You can import the dataset using the code below.

\begin{Shaded}
\begin{Highlighting}[]
\NormalTok{d }\OtherTok{\textless{}{-}} \FunctionTok{read.csv}\NormalTok{(}\StringTok{"./honeyproduction.csv"}\NormalTok{,}\AttributeTok{header=}\ConstantTok{TRUE}\NormalTok{)}
\end{Highlighting}
\end{Shaded}

\begin{enumerate}
\def\labelenumi{\arabic{enumi}.}
\tightlist
\item
  Visualize the total production of honey (\textbf{totalprod}) across
  years (\textbf{year}) by state (\textbf{state}). Use color to
  highlight the west coast (Washington, Oregon, and California) with a
  different color used for each west coast state.
\end{enumerate}

\begin{itemize}
\tightlist
\item
  \textbf{Hint 1}: I'm not asking for a specific kind of plot, just one
  that does the preceding. But if you're trying to visualize change over
  time, a bar chart is likely not going to be the best choice.
\item
  \textbf{Hint 2}: To get each state to be a different color you should
  either map state to color (for your layer that adds the west coast
  colors) or use the
  \href{https://yutannihilation.github.io/gghighlight/index.html}{gghighlight}
  package.
\end{itemize}

\begin{Shaded}
\begin{Highlighting}[]
\NormalTok{p1 }\OtherTok{\textless{}{-}} \FunctionTok{ggplot}\NormalTok{(d, }\FunctionTok{aes}\NormalTok{(}\AttributeTok{x =}\NormalTok{ year, }\AttributeTok{y =}\NormalTok{ totalprod))}

\NormalTok{p1 }\OtherTok{\textless{}{-}}
\NormalTok{  p1 }\SpecialCharTok{+} \FunctionTok{geom\_line}\NormalTok{(}\FunctionTok{aes}\NormalTok{(}\AttributeTok{color =}\NormalTok{ state)) }\SpecialCharTok{+}
   \FunctionTok{scale\_color\_manual}\NormalTok{(}\AttributeTok{values =} \FunctionTok{c}\NormalTok{(}\StringTok{"CA"} \OtherTok{=} \StringTok{"red"}\NormalTok{, }\StringTok{"OR"} \OtherTok{=} \StringTok{"yellow"}\NormalTok{, }
                                \StringTok{"WA"} \OtherTok{=} \StringTok{"blue"}\NormalTok{)) }\SpecialCharTok{+}
  \FunctionTok{facet\_wrap}\NormalTok{(}\SpecialCharTok{\textasciitilde{}}\NormalTok{state, }\AttributeTok{ncol =} \DecValTok{8}\NormalTok{) }\SpecialCharTok{+} 
  \FunctionTok{theme\_light}\NormalTok{()}\SpecialCharTok{+}
  \FunctionTok{scale\_y\_continuous}\NormalTok{(}\AttributeTok{labels =}\NormalTok{ scales}\SpecialCharTok{::}\NormalTok{comma)}

\FunctionTok{options}\NormalTok{(}\AttributeTok{scipen=}\DecValTok{999}\NormalTok{)}

\NormalTok{p1}
\end{Highlighting}
\end{Shaded}

\begin{enumerate}
\def\labelenumi{\arabic{enumi}.}
\setcounter{enumi}{1}
\tightlist
\item
  Reproduce the plot according three different kinds of color blindness
  using the \texttt{cvd\_grid} package from the \texttt{colorblindr}
  package.
\end{enumerate}

\begin{Shaded}
\begin{Highlighting}[]
\CommentTok{\#remotes::install\_github("clauswilke/colorblindr")}

\FunctionTok{library}\NormalTok{(colorblindr)}
\FunctionTok{cvd\_grid}\NormalTok{(p1)}
\end{Highlighting}
\end{Shaded}

\begin{enumerate}
\def\labelenumi{\arabic{enumi}.}
\setcounter{enumi}{2}
\tightlist
\item
  Reproduce the plot using a color blind safe palette of your choice.
\end{enumerate}

\begin{Shaded}
\begin{Highlighting}[]
\NormalTok{cbPalette }\OtherTok{\textless{}{-}} \FunctionTok{c}\NormalTok{(}\StringTok{"\#999999"}\NormalTok{, }\StringTok{"\#E69F00"}\NormalTok{, }\StringTok{"\#56B4E9"}\NormalTok{, }\StringTok{"\#009E73"}\NormalTok{, }\StringTok{"\#F0E442"}\NormalTok{, }\StringTok{"\#0072B2"}\NormalTok{, }\StringTok{"\#D55E00"}\NormalTok{, }\StringTok{"\#CC79A7"}\NormalTok{)}

\NormalTok{p3 }\OtherTok{\textless{}{-}}\NormalTok{ p1 }\SpecialCharTok{+} 
  \FunctionTok{geom\_line}\NormalTok{(}\FunctionTok{aes}\NormalTok{(}\AttributeTok{color =}\NormalTok{ state)) }\SpecialCharTok{+}
  \FunctionTok{scale\_color\_manual}\NormalTok{(}\AttributeTok{values =} \FunctionTok{c}\NormalTok{(}\StringTok{"CA"} \OtherTok{=} \StringTok{"\#0072B2"}\NormalTok{, }\StringTok{"OR"} \OtherTok{=} \StringTok{"\#D55E00"}\NormalTok{, }
                                \StringTok{"WA"} \OtherTok{=} \StringTok{"\#009E73"}\NormalTok{)) }\SpecialCharTok{+}
  \FunctionTok{facet\_wrap}\NormalTok{(}\SpecialCharTok{\textasciitilde{}}\NormalTok{state, }\AttributeTok{ncol =} \DecValTok{8}\NormalTok{) }\SpecialCharTok{+} 
  \FunctionTok{theme\_light}\NormalTok{()}

\NormalTok{p3}
\end{Highlighting}
\end{Shaded}

\begin{enumerate}
\def\labelenumi{\arabic{enumi}.}
\setcounter{enumi}{3}
\tightlist
\item
  Download the file \textbf{`us census bureau regions and
  divisions.csv'} from the course website denoting the region and
  division of each state.
\end{enumerate}

\begin{itemize}
\tightlist
\item
  Join the file with your honey file.
\item
  Produce a bar plot displaying the average honey for each state
  (collapsing across years).
\item
  Use color to highlight the region of the country the state is from.
\item
  Note patterns you notice.
\end{itemize}

The plot should look like similar to the following plot (see the pdf).

\begin{Shaded}
\begin{Highlighting}[]
\NormalTok{df }\OtherTok{\textless{}{-}} \FunctionTok{read.csv}\NormalTok{(}\StringTok{"./us census bureau regions and divisions.csv"}\NormalTok{,}\AttributeTok{header=}\ConstantTok{TRUE}\NormalTok{)}

\NormalTok{df }\OtherTok{\textless{}{-}}\NormalTok{ df }\SpecialCharTok{|\textgreater{}}  \FunctionTok{rename}\NormalTok{(}\StringTok{"state"} \OtherTok{=} \StringTok{"State.Code"}\NormalTok{)}

\NormalTok{df }\OtherTok{\textless{}{-}} \FunctionTok{left\_join}\NormalTok{(d, df)}


\NormalTok{df2 }\OtherTok{\textless{}{-}}\NormalTok{ df }\SpecialCharTok{|\textgreater{}} 
    \FunctionTok{group\_by}\NormalTok{(State, Region) }\SpecialCharTok{|\textgreater{}} 
    \FunctionTok{summarise}\NormalTok{(}\AttributeTok{avg =} \FunctionTok{mean}\NormalTok{(totalprod)) }\SpecialCharTok{|\textgreater{}} 
  \FunctionTok{arrange}\NormalTok{(}\FunctionTok{desc}\NormalTok{(avg))}

\NormalTok{p4 }\OtherTok{\textless{}{-}}\NormalTok{ df2 }\SpecialCharTok{|\textgreater{}} 
  \FunctionTok{ggplot}\NormalTok{(}\FunctionTok{aes}\NormalTok{(}\AttributeTok{x=}\FunctionTok{reorder}\NormalTok{(State, avg), }\AttributeTok{y=}\NormalTok{avg, }\AttributeTok{fill=}\NormalTok{Region)) }\SpecialCharTok{+}
  \FunctionTok{geom\_col}\NormalTok{() }\SpecialCharTok{+}
  \FunctionTok{coord\_flip}\NormalTok{()}\SpecialCharTok{+}
  \FunctionTok{scale\_fill\_viridis}\NormalTok{(}\AttributeTok{discrete =} \ConstantTok{TRUE}\NormalTok{, }\AttributeTok{option =} \StringTok{"B"}\NormalTok{) }\SpecialCharTok{+}
  \FunctionTok{xlab}\NormalTok{(}\StringTok{" "}\NormalTok{)}\SpecialCharTok{+}
  \FunctionTok{ylab}\NormalTok{(}\StringTok{"Average Honey Production (lbs)"}\NormalTok{)}\SpecialCharTok{+}
  \FunctionTok{scale\_y\_continuous}\NormalTok{(}\AttributeTok{labels =}\NormalTok{ scales}\SpecialCharTok{::}\NormalTok{comma)}\SpecialCharTok{+}
  \FunctionTok{theme\_light}\NormalTok{()}


\NormalTok{p4}
\end{Highlighting}
\end{Shaded}

\begin{enumerate}
\def\labelenumi{\arabic{enumi}.}
\setcounter{enumi}{4}
\tightlist
\item
  Create a heatmap displaying the average honey production across years
  by \emph{region} (averaging across states within region). The plot
  should look like similar to the following plot (see the pdf).
\end{enumerate}

\begin{Shaded}
\begin{Highlighting}[]
\NormalTok{df3 }\OtherTok{\textless{}{-}}\NormalTok{ df }\SpecialCharTok{|\textgreater{}} 
    \FunctionTok{group\_by}\NormalTok{(Region, year) }\SpecialCharTok{|\textgreater{}} 
    \FunctionTok{summarise}\NormalTok{(}\AttributeTok{avg =} \FunctionTok{mean}\NormalTok{(totalprod)) }\SpecialCharTok{|\textgreater{}} 
  \FunctionTok{arrange}\NormalTok{(}\FunctionTok{desc}\NormalTok{(avg))}

\NormalTok{p5 }\OtherTok{\textless{}{-}}\NormalTok{ df3 }\SpecialCharTok{|\textgreater{}} 
  \FunctionTok{ggplot}\NormalTok{(}\FunctionTok{aes}\NormalTok{(}\AttributeTok{x=}\NormalTok{year, }\AttributeTok{y=}\NormalTok{Region, }\AttributeTok{fill=}\NormalTok{avg}\SpecialCharTok{/}\FloatTok{1e6}\NormalTok{)) }\SpecialCharTok{+}
  \FunctionTok{geom\_tile}\NormalTok{() }\SpecialCharTok{+}
  \FunctionTok{scale\_fill\_viridis}\NormalTok{(}\AttributeTok{discrete=}\ConstantTok{FALSE}\NormalTok{, }\AttributeTok{option=}\StringTok{"A"}\NormalTok{,}
                     \AttributeTok{name=}\StringTok{"Honey Production }\SpecialCharTok{\textbackslash{}n}\StringTok{(Millions of lbs)}\SpecialCharTok{\textbackslash{}n}\StringTok{"}\NormalTok{) }\SpecialCharTok{+}
  \FunctionTok{xlab}\NormalTok{(}\StringTok{"Year"}\NormalTok{)}\SpecialCharTok{+}
  \FunctionTok{ylab}\NormalTok{(}\StringTok{"Region"}\NormalTok{)}\SpecialCharTok{+}
  \FunctionTok{theme}\NormalTok{(}\AttributeTok{legend.position=}\StringTok{\textquotesingle{}top\textquotesingle{}}\NormalTok{, }\AttributeTok{panel.grid.major =} \FunctionTok{element\_blank}\NormalTok{(), }\AttributeTok{panel.grid.minor =} \FunctionTok{element\_blank}\NormalTok{(),}
\AttributeTok{panel.background =} \FunctionTok{element\_blank}\NormalTok{())}

\NormalTok{p5}
\end{Highlighting}
\end{Shaded}

\begin{enumerate}
\def\labelenumi{\arabic{enumi}.}
\setcounter{enumi}{5}
\tightlist
\item
  Create at least one more plot of your choosing using color to
  distinguish, represent data values, or highlight. If you are
  interested in producing maps, I suggest grabbing a simple features
  data frame of the US using the Albers projection by doing the
  following:
\end{enumerate}

\begin{Shaded}
\begin{Highlighting}[]
\NormalTok{remotes}\SpecialCharTok{::}\FunctionTok{install\_github}\NormalTok{(}\StringTok{"hrbrmstr/albersusa"}\NormalTok{)}
\FunctionTok{library}\NormalTok{(albersusa)}
\NormalTok{us }\OtherTok{\textless{}{-}} \FunctionTok{usa\_sf}\NormalTok{()}
\end{Highlighting}
\end{Shaded}

You can then join your honey data with this dataset. We'll talk about
geographic data more later in the course, but it is pretty easy to work
with. For example, you could use the data frame above to create a map of
the US with:

\begin{Shaded}
\begin{Highlighting}[]
\FunctionTok{ggplot}\NormalTok{(us) }\SpecialCharTok{+}
  \FunctionTok{geom\_sf}\NormalTok{()}
\end{Highlighting}
\end{Shaded}

You will likely get a few warnings but they are most likely ignorable.
You could also theme it more from here, but if you join it with your
honey data, you should be able to \texttt{fill} and/or \texttt{facet} by
specific variables.

Note - this is a little trickier than it initially seems because you can
``lose'' states in your map if they don't have any data (there are only
44 states represented in your honey dataset). You should still plot all
states, but just have them not be filled according to your fill scale if
there is no data.

For instance, below is a plot created based on this data for inspiration
(see the pdf).

Example 1:

\begin{Shaded}
\begin{Highlighting}[]
\NormalTok{honey\_subset }\OtherTok{\textless{}{-}}\NormalTok{ df }\SpecialCharTok{\%\textgreater{}\%}
  \FunctionTok{select}\NormalTok{(}\StringTok{"iso\_3166\_2"} \OtherTok{=} \StringTok{"state"}\NormalTok{, year, totalprod)}

\NormalTok{full\_set }\OtherTok{\textless{}{-}} \FunctionTok{expand.grid}\NormalTok{(}\AttributeTok{iso\_3166\_2 =} \FunctionTok{unique}\NormalTok{(us}\SpecialCharTok{$}\NormalTok{iso\_3166\_2),}
\AttributeTok{year =} \DecValTok{1998}\SpecialCharTok{:}\DecValTok{2012}\NormalTok{)}

\NormalTok{honey\_subset }\OtherTok{\textless{}{-}} \FunctionTok{left\_join}\NormalTok{(full\_set, honey\_subset)}

\NormalTok{honey\_geo }\OtherTok{\textless{}{-}} \FunctionTok{left\_join}\NormalTok{(us, honey\_subset)}

\NormalTok{eg }\OtherTok{\textless{}{-}} \FunctionTok{ggplot}\NormalTok{(honey\_geo) }\SpecialCharTok{+}
  \FunctionTok{geom\_sf}\NormalTok{(}\FunctionTok{aes}\NormalTok{(}\AttributeTok{fill =}\NormalTok{ totalprod}\SpecialCharTok{/}\FloatTok{1e6}\NormalTok{)) }\SpecialCharTok{+}
  \FunctionTok{facet\_wrap}\NormalTok{(}\SpecialCharTok{\textasciitilde{}}\NormalTok{year) }\SpecialCharTok{+}
\NormalTok{  colorspace}\SpecialCharTok{::}\FunctionTok{scale\_fill\_continuous\_sequential}\NormalTok{(}\AttributeTok{palette =} \StringTok{"Purples"}\NormalTok{, }
                                               \AttributeTok{na.value =} \StringTok{"white"}\NormalTok{,}
                                               \AttributeTok{name =} \StringTok{"Honey Production}\SpecialCharTok{\textbackslash{}n}\StringTok{(Millions of lbs)}\SpecialCharTok{\textbackslash{}n}\StringTok{"}
\NormalTok{                                               ) }\SpecialCharTok{+}
  \FunctionTok{theme\_minimal}\NormalTok{() }\SpecialCharTok{+}
  \FunctionTok{theme}\NormalTok{(}\AttributeTok{legend.direction =} \StringTok{"horizontal"}\NormalTok{,}
        \AttributeTok{legend.position =} \StringTok{"bottom"}\NormalTok{,}
        \AttributeTok{legend.key.size =} \FunctionTok{unit}\NormalTok{(}\DecValTok{2}\NormalTok{, }\StringTok{\textquotesingle{}cm\textquotesingle{}}\NormalTok{),}
        \AttributeTok{legend.key.height =} \FunctionTok{unit}\NormalTok{(.}\DecValTok{5}\NormalTok{,}\StringTok{"cm"}\NormalTok{),}
        \AttributeTok{axis.text.x =} \FunctionTok{element\_text}\NormalTok{(}\AttributeTok{size =} \DecValTok{5}\NormalTok{))}
\end{Highlighting}
\end{Shaded}

Practice 1

\begin{Shaded}
\begin{Highlighting}[]
\NormalTok{devtools}\SpecialCharTok{::}\FunctionTok{install\_github}\NormalTok{(}\StringTok{\textquotesingle{}thomasp85/gganimate\textquotesingle{}}\NormalTok{)}

\NormalTok{prac }\OtherTok{\textless{}{-}} \FunctionTok{ggplot}\NormalTok{(df, }\FunctionTok{aes}\NormalTok{(yieldpercol,totalprod, }\AttributeTok{size=}\NormalTok{priceperlb, }\AttributeTok{colour=}\NormalTok{state)) }\SpecialCharTok{+}
  \FunctionTok{geom\_point}\NormalTok{(}\AttributeTok{alpha=}\FloatTok{0.6}\NormalTok{, }\AttributeTok{show.legend=}\ConstantTok{FALSE}\NormalTok{) }\SpecialCharTok{+}
  \FunctionTok{scale\_size}\NormalTok{(}\AttributeTok{range=}\FunctionTok{c}\NormalTok{(}\DecValTok{2}\NormalTok{,}\DecValTok{12}\NormalTok{))}\SpecialCharTok{+}
  \FunctionTok{scale\_x\_log10}\NormalTok{()}\SpecialCharTok{+}
  \FunctionTok{facet\_wrap}\NormalTok{(}\SpecialCharTok{\textasciitilde{}}\NormalTok{Region)}\SpecialCharTok{+}
  \FunctionTok{labs}\NormalTok{(}\AttributeTok{title=}\StringTok{"Year: \{frame\_time\}"}\NormalTok{, }\AttributeTok{x=}\StringTok{"Pound of Honey Per Colony"}\NormalTok{, }\AttributeTok{y=}\StringTok{"Total Pounds of Honey Production"}\NormalTok{)}\SpecialCharTok{+}
  \FunctionTok{transition\_time}\NormalTok{(year)}\SpecialCharTok{+}
  \FunctionTok{scale\_y\_continuous}\NormalTok{(}\AttributeTok{labels =}\NormalTok{ scales}\SpecialCharTok{::}\NormalTok{comma)}\SpecialCharTok{+}
  \FunctionTok{ease\_aes}\NormalTok{(}\StringTok{\textquotesingle{}linear\textquotesingle{}}\NormalTok{)}
  
\FunctionTok{library}\NormalTok{(gifski)}
\FunctionTok{animate}\NormalTok{(prac, }\AttributeTok{duration =} \DecValTok{7}\NormalTok{, }\AttributeTok{fps =} \DecValTok{20}\NormalTok{, }\AttributeTok{width =} \DecValTok{400}\NormalTok{, }\AttributeTok{height =} \DecValTok{400}\NormalTok{, }\AttributeTok{renderer =} \FunctionTok{gifski\_renderer}\NormalTok{())}
\FunctionTok{anim\_save}\NormalTok{(}\StringTok{"output.gif"}\NormalTok{)}
\end{Highlighting}
\end{Shaded}


\end{document}
